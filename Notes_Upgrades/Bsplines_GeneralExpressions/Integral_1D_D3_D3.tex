%%%%%%%%%%%%%%%%%%%%%%%%%%%%%%%%%%%%%%%%%%%%%%%%%%%%%%%%%%%%%%%%%%%%%%
% LaTeX Example: Project Report
%
% Source: http://www.howtotex.com
%
% Feel free to distribute this example, but please keep the referral
% to howtotex.com
% Date: March 2011
%
%%%%%%%%%%%%%%%%%%%%%%%%%%%%%%%%%%%%%%%%%%%%%%%%%%%%%%%%%%%%%%%%%%%%%%
% How to use writeLaTeX:
%
% You edit the source code here on the left, and the preview on the
% right shows you the result within a few seconds.
%
% Bookmark this page and share the URL with your co-authors. They can
% edit at the same time!
%
% You can upload figures, bibliographies, custom classes and
% styles using the files menu.
%
% If you're new to LaTeX, the wikibook is a great place to start:
% http://en.wikibooks.org/wiki/LaTeX
%
%%%%%%%%%%%%%%%%%%%%%%%%%%%%%%%%%%%%%%%%%%%%%%%%%%%%%%%%%%%%%%%%%%%%%%
% Edit the title below to update the display in My Documents
%\title{Project Report}
%
%%% Preamble
\documentclass[paper=a4, fontsize=11pt]{scrartcl}
\usepackage[T1]{fontenc}
\usepackage{fourier}

\usepackage[english]{babel}															% English language/hyphenation
\usepackage[protrusion=true,expansion=true]{microtype}
\usepackage{amsmath,amsfonts,amsthm} % Math packages
\usepackage[pdftex]{graphicx}
\usepackage{url}
\usepackage{caption}
\usepackage[top=1in, bottom=1in, left=1in, right=1in]{geometry}
\usepackage{subcaption}  % For placing two subfigures side by side

%%% Custom sectioning
\usepackage{sectsty}
\usepackage{multirow}
\allsectionsfont{\centering \normalfont\scshape}

%%% Inserting landscape pages
\usepackage{pdflscape}


%%% Custom headers/footers (fancyhdr package)
\usepackage{fancyhdr}
\pagestyle{fancyplain}
\fancyhead{}											% No page header
\fancyfoot[L]{}											% Empty
\fancyfoot[C]{}											% Empty
\fancyfoot[R]{\thepage}									% Pagenumbering
\renewcommand{\headrulewidth}{0pt}			% Remove header underlines
\renewcommand{\footrulewidth}{0pt}				% Remove footer underlines
\setlength{\headheight}{13.6pt}
\setlength{\fboxrule}{1mm}
\setlength{\fboxsep}{5mm}

%%% Equation and float numbering
\numberwithin{equation}{section}		% Equationnumbering: section.eq#
\numberwithin{figure}{section}			% Figurenumbering: section.fig#
\numberwithin{table}{section}				% Tablenumbering: section.tab#


%%% Maketitle metadata
\newcommand{\horrule}[1]{\rule{\linewidth}{#1}} 	% Horizontal rule

\title{
		%\vspace{-1in}
		\usefont{OT1}{bch}{b}{n}
		\horrule{0.5pt} \\[0.2cm]
		\LARGE Deriving an analytical expression for the local extrema of a bivariate B-spline of degree 2\\
        -\\
        \normalsize Another tool for characterising sawtooth crash precursors
		\horrule{2pt} \\[0.3cm]
}
\author{D. VEZINET}
\date{\today}

% Graphics path
\graphicspath{ {./} }

%%% Begin document
\begin{document}
\maketitle

\tableofcontents

\newpage
\section{General expression of the surface and gradient}


\begin{figure}[hbtp]
    \centering
    \includegraphics[scale=0.35]{Fig_Grid.png}
    \caption{\small Grid element on which we want to check the existence of a point of null gradient}
    \label{Fig:Grid}
\end{figure}


On the mesh element illustrated in \ref{Fig:Grid} (disregarding boundary effects, i.e.: assuming no basis function is missing) a bivariate B-spline of degree 2 can be written:
$$
\begin{array}{ll}
F(x,y) &= \sum_{i=0}^{9}C_{i}F_i(x,y)\\
&=
\begin{array}{ll}
& C_{00}f_{x0..3}(x)f_{y0..3}(y) + C_{01}f_{x1..4}(x)f_{y0..3}(y) + C_{02}f_{x2..5}(x)f_{y0..3}(y)\\
+& C_{10}f_{x0..3}(x)f_{y1..4}(y) + C_{11}f_{x1..4}(x)f_{y1..4}(y) + C_{12}f_{x2..5}(x)f_{y1..4}(y)\\
+& C_{20}f_{x0..3}(x)f_{y2..5}(y) + C_{21}f_{x1..4}(x)f_{y2..5}(y) + C_{22}f_{x2..5}(x)f_{y2..5}(y)
\end{array}
\end{array}
$$

Where $f_{x0..3} =
\left\{
\begin{array}{lll}
\frac{(x-x0)^2}{(x2-x0)(x1-x0)} & \text{ ,  if  } & x \in [x0,x1]\\
\frac{(x-x0)(x2-x)}{(x2-x0)(x2-x1)} + \frac{(x-x1)(x3-x)}{(x2-x1)(x3-x1)} & \text{ ,  if  } & x \in [x1,x2]\\
\frac{(x3-x)^2}{(x3-x2)(x3-x1)} & \text{ ,  if  } & x \in [x2,x3]
\end{array}
\right.
$

So $\frac{\partial f_{x0..3}}{\partial x} =
\left\{
\begin{array}{lll}
\frac{2(x-x0)}{(x2-x0)(x1-x0)} & \text{ ,  if  } & x \in [x0,x1]\\
\frac{-2x+(x0+x2)}{(x2-x0)(x2-x1)} + \frac{-2x+(x1+x3)}{(x2-x1)(x3-x1)} & \text{ ,  if  } & x \in [x1,x2]\\
\frac{-2(x3-x)}{(x3-x2)(x3-x1)} & \text{ ,  if  } & x \in [x2,x3]
\end{array}
\right.
$

We want to know for which points inside the mesh element of interest (i.e. $(x,y) \in [a2,a3]\times[b2,b3]$) we have:
$$
\begin{array}{ccc}
\underline{\nabla}F = 0 &\Leftrightarrow &
\left \{
\begin{array}{llc}
\sum_{i=0}^9 C_i f_{y...}(y)\frac{\partial f_{x...}}{\partial x}(x) = 0 & (1)\\
\sum_{i=0}^9 C_i f_{x...}(x)\frac{\partial f_{y...}}{\partial y}(y) = 0 & (2)
\end{array}
\right.
\end{array}
$$

We will treat case (1) in the following and deduce the similar expression for case (2).

%\frac{(x-x0)^2}{(x2-x0)(x1-x0)}

\newpage
\begin{landscape}
\section{Calculations}

\paragraph{\textbf{Trivial case}} If all $C_i$ are zero then the gradient is zero everywhere on the mesh element.

\paragraph{\textbf{Non-trivial case for (1)}} ... and assuming that all the knots are different
$$
\begin{array}{lll}
0 & = & \frac{(y3-y)^2}{(y3-y2)(y3-y1)} \left( C_{00}\frac{-2(x3-x)}{(x3-x2)(x3-x1)} + C_{01}\left( \frac{-2x+(x1+x3)}{(x3-x1)(x3-x2)} + \frac{-2x+(x2+x4)}{(x3-x2)(x4-x2)} \right) + C_{02}\frac{2(x-x2)}{(x4-x2)(x3-x2)}  \right)\\
& & + \left( \frac{(y-y1)(y3-y)}{(y3-y1)(y3-y2)} + \frac{(y-y2)(y4-y)}{(y3-y2)(y4-y2)} \right) \left( C_{10}\frac{-2(x3-x)}{(x3-x2)(x3-x1)} + C_{11}\left( \frac{-2x+(x1+x3)}{(x3-x1)(x3-x2)} + \frac{-2x+(x2+x4)}{(x3-x2)(x4-x2)} \right) + C_{12}\frac{2(x-x2)}{(x4-x2)(x3-x2)} \right)\\
& & + \frac{(y-y2)^2}{(y4-y2)(y3-y2)} \left( C_{20}\frac{-2(x3-x)}{(x3-x2)(x3-x1)} + C_{21}\left( \frac{-2x+(x1+x3)}{(x3-x1)(x3-x2)} + \frac{-2x+(x2+x4)}{(x3-x2)(x4-x2)} \right) + C_{22}\frac{2(x-x2)}{(x4-x2)(x3-x2)} \right)\\

& = & \frac{(y3-y)^2}{(y3-y1)} \left( C_{00}\frac{-2(x3-x)}{(x3-x1)} + C_{01}\left( \frac{-2x+(x1+x3)}{(x3-x1)} + \frac{-2x+(x2+x4)}{(x4-x2)} \right) + C_{02}\frac{2(x-x2)}{(x4-x2)}  \right)\\
& & + \left( \frac{(y-y1)(y3-y)}{(y3-y1)} + \frac{(y-y2)(y4-y)}{(y4-y2)} \right) \left( C_{10}\frac{-2(x3-x)}{(x3-x1)} + C_{11}\left( \frac{-2x+(x1+x3)}{(x3-x1)} + \frac{-2x+(x2+x4)}{(x4-x2)} \right) + C_{12}\frac{2(x-x2)}{(x4-x2)} \right)\\
& & + \frac{(y-y2)^2}{(y4-y2)} \left( C_{20}\frac{-2(x3-x)}{(x3-x1)} + C_{21}\left( \frac{-2x+(x1+x3)}{(x3-x1)} + \frac{-2x+(x2+x4)}{(x4-x2)} \right) + C_{22}\frac{2(x-x2)}{(x4-x2)} \right)\\

& = & \frac{(y3-y)^2}{(y3-y1)} \left( x\left( \frac{2C_{00}}{(x3-x1)} + \frac{-2C_{01}}{(x3-x1)} + \frac{-2C_{01}}{(x4-x2)} + \frac{2C_{02}}{(x4-x2)} \right) +  \frac{-2x3C_{00}}{(x3-x1)} + \frac{C_{01}(x1+x3)}{(x3-x1)} + \frac{C_{01}(x2+x4)}{(x4-x2)} + \frac{-2x2C_{02}}{(x4-x2)} \right)\\
& & + \left( \frac{(y-y1)(y3-y)}{(y3-y1)} + \frac{(y-y2)(y4-y)}{(y4-y2)} \right) \left( x\left( \frac{2C_{10}}{(x3-x1)} + \frac{-2C_{11}}{(x3-x1)} + \frac{-2C_{11}}{(x4-x2)} + \frac{2C_{12}}{(x4-x2)} \right) +  \frac{-2x3C_{10}}{(x3-x1)} + \frac{C_{11}(x1+x3)}{(x3-x1)} + \frac{C_{11}(x2+x4)}{(x4-x2)} + \frac{-2x2C_{12}}{(x4-x2)} \right)\\
& & + \frac{(y-y2)^2}{(y4-y2)} \left( x\left( \frac{2C_{20}}{(x3-x1)} + \frac{-2C_{21}}{(x3-x1)} + \frac{-2C_{21}}{(x4-x2)} + \frac{2C_{22}}{(x4-x2)} \right) +  \frac{-2x3C_{20}}{(x3-x1)} + \frac{C_{21}(x1+x3)}{(x3-x1)} + \frac{C_{21}(x2+x4)}{(x4-x2)} + \frac{-2x2C_{22}}{(x4-x2)} \right)
\end{array}
$$

So, by introducing $\left \{ \begin{array}{l} A_{x,i} = \frac{2(C_{i0}-C_{i1})}{(x3-x1)} + \frac{2(C_{i2}-C_{i1})}{(x4-x2)}\\ B_{x,i} = \frac{C_{i1}(x1+x3)-2x3C_{i0}}{(x3-x1)} + \frac{C_{i1}(x2+x4)-2x2C_{i2}}{(x4-x2)} \end{array} \right.$, we can write:
$$
\begin{array}{lll}
0 & = & \frac{(y3-y)^2}{(y3-y1)} \left( xA_{x,0} + B_{x,0} \right) + \left( \frac{(y-y1)(y3-y)}{(y3-y1)} + \frac{(y-y2)(y4-y)}{(y4-y2)} \right) \left( xA_{x,1} + B_{x,1} \right) + \frac{(y-y2)^2}{(y4-y2)} \left( xA_{x,2} + B_{x,2} \right)\\
& = & y^2 \left( \frac{xA_{x,0} + B_{x,0}}{y3-y1} - \frac{xA_{x,1} + B_{x,1}}{y3-y1} - \frac{xA_{x,1} + B_{x,1}}{y4-y2} + \frac{xA_{x,2} + B_{x,2}}{y4-y2} \right)\\
& & + y\left( -2y3\frac{xA_{x,0} + B_{x,0}}{y3-y1} + \frac{y3+y1}{y3-y1}\left( xA_{x,1} + B_{x,1} \right) + \frac{y4+y2}{y4-y2}\left( xA_{x,1} + B_{x,1} \right) - 2y2\frac{xA_{x,2} + B_{x,2}}{y4-y2} \right)\\
& & + y3^2\frac{xA_{x,0} + B_{x,0}}{y3-y1} - y3y1\frac{xA_{x,1} + B_{x,1}}{y3-y1} - y4y2\frac{xA_{x,1} + B_{x,1}}{y4-y2} + y2^2\frac{xA_{x,2} + B_{x,2}}{y4-y2}\\

& = & y^2 \left( \frac{x\left(A_{x,0}-A_{x,1}\right) + B_{x,0}-B_{x,1}}{y3-y1} + \frac{x\left(A_{x,2}-A_{x,1}\right) + B_{x,2}-B_{x,1}}{y4-y2} \right)\\
& & + y\left( \frac{(y3+y1)\left( xA_{x,1} + B_{x,1} \right) - 2y3\left(xA_{x,0} + B_{x,0}\right)}{y3-y1} + \frac{(y4+y2)\left( xA_{x,1} + B_{x,1} \right) - 2y2\left(xA_{x,2} + B_{x,2}\right)}{y4-y2} \right)\\
& & + \frac{y3^2\left(xA_{x,0} + B_{x,0}\right) - y3y1\left(xA_{x,1} + B_{x,1}\right)}{y3-y1} + \frac{y2^2\left(xA_{x,2} + B_{x,2}\right) - y4y2\left(xA_{x,1} + B_{x,1}\right)}{y4-y2}
\end{array}
$$

\end{landscape}

\newpage
\section{Overview of the reduced set of equations}
Hence, we can write that a point with gradient zero will have:
$$
\left \{
\begin{array}{l}
y^2 P_2(x) + yP_1(x) + P_0(x) = 0\\
x^2 Q_2(y) + xQ_1(y) + Q_0(y) = 0
\end{array}
\right.
$$

Where:
$$
\left \{
\begin{array}{ll}
P_2(x) = x\left( \frac{A_{x,0}-A_{x,1}}{y3-y1} + \frac{A_{x,2}-A_{x,1}}{y4-y2} \right) + \frac{B_{x,0}-B_{x,1}}{y3-y1} + \frac{B_{x,2}-B_{x,1}}{y4-y2}  & = xa_{P2} + b_{P2}\\
P_1(x) = x\left( \frac{(y3+y1)A_{x,1} - 2y3A_{x,0}}{y3-y1} + \frac{(y4+y2)A_{x,1} - 2y2A_{x,2}}{y4-y2} \right) + \frac{(y3+y1)B_{x,1} - 2y3B_{x,0}}{y3-y1} + \frac{(y4+y2)B_{x,1} - 2y2B_{x,2}}{y4-y2}  & = xa_{P1} + b_{P1}\\
P_0(x) = x\left( \frac{y3^2A_{x,0} - y3y1xA_{x,1}}{y3-y1} + \frac{y2^2A_{x,2} - y4y2A_{x,1}}{y4-y2} \right) + \frac{y3^2B_{x,0} - y3y1xB_{x,1}}{y3-y1} + \frac{y2^2B_{x,2} - y4y2B_{x,1}}{y4-y2}  & = xa_{P0} + b_{P0}
\end{array}
\right.
$$

And:
$$
\left \{
\begin{array}{ll}
Q_2(y) = y\left( \frac{A_{y,0}-A_{y,1}}{x3-x1} + \frac{A_{y,2}-A_{y,1}}{x4-x2} \right) + \frac{B_{y,0}-B_{y,1}}{x3-x1} + \frac{B_{y,2}-B_{y,1}}{x4-x2}  & = ya_{Q2} + b_{Q2}\\
Q_1(y) = y\left( \frac{(x3+x1)A_{y,1} - 2x3A_{y,0}}{x3-x1} + \frac{(x4+x2)A_{y,1} - 2x2A_{y,2}}{x4-x2} \right) + \frac{(x3+x1)B_{y,1} - 2x3B_{y,0}}{x3-x1} + \frac{(x4+x2)B_{y,1} - 2x2B_{y,2}}{x4-x2}  & = ya_{Q1} + b_{Q1}\\
Q_0(y) = y\left( \frac{x3^2A_{y,0} - x3x1xA_{y,1}}{x3-x1} + \frac{x2^2A_{y,2} - x4x2A_{y,1}}{x4-x2} \right) + \frac{x3^2B_{y,0} - x3x1xB_{y,1}}{x3-x1} + \frac{x2^2B_{y,2} - x4x2B_{y,1}}{x4-x2}  & = ya_{Q0} + b_{Q0}
\end{array}
\right.
$$

In the following, we will explore the various cases of this set of equations, keeping in mind that we are looking for a solution in $[x2,x3]\times[y2,y3]$

\newpage
\section{Cases of the reduced set of equations}

\subsection{$P_2(x) = 0$}

Necessarily, if $P_2(x)=0$ then:
$$
x\left( \frac{A_{x,0}-A_{x,1}}{y3-y1} + \frac{A_{x,2}-A_{x,1}}{y4-y2} \right) = -\frac{B_{x,0}-B_{x,1}}{y3-y1} + \frac{B_{x,2}-B_{x,1}}{y4-y2}\\
$$

\paragraph{\textbf{Case 01}} $\frac{A_{x,0}-A_{x,1}}{y3-y1} + \frac{A_{x,2}-A_{x,1}}{y4-y2} = 0$\\
%Then $\frac{2}{y3-y1}\left( \frac{(C_{00}-C_{01})}{(x3-x1)} + \frac{(C_{02}-C_{01})}{(x4-x2)} - \frac{(C_{10}-C_{11})}{(x3-x1)} + \frac{(C_{12}-C_{11})}{(x4-x2)} \right)
%+ \frac{2}{y4-y2}\left( \frac{(C_{20}-C_{21})}{(x3-x1)} + \frac{(C_{22}-C_{21})}{(x4-x2)} - \frac{(C_{10}-C_{11})}{(x3-x1)} + \frac{(C_{12}-C_{11})}{(x4-x2)} \right) = 0$
Then:
\begin{enumerate}
\item If $\frac{B_{x,0}-B_{x,1}}{y3-y1} + \frac{B_{x,2}-B_{x,1}}{y4-y2} \neq 0$, no possible solution
\item If $\frac{B_{x,0}-B_{x,1}}{y3-y1} + \frac{B_{x,2}-B_{x,1}}{y4-y2} = 0$, all x are solution
\end{enumerate}

Thus, some solutions are accessible in $[x2,x3]\times[y2,y3]$ in the second case.

\paragraph{\textbf{Case 02}} $\frac{A_{x,0}-A_{x,1}}{y3-y1} + \frac{A_{x,2}-A_{x,1}}{y4-y2} \neq 0$\\
Then $x_0 = -\frac{\frac{B_{x,0}-B_{x,1}}{y3-y1} + \frac{B_{x,2}-B_{x,1}}{y4-y2}}{\frac{A_{x,0}-A_{x,1}}{y3-y1} + \frac{A_{x,2}-A_{x,1}}{y4-y2}}$\\
Thus, a solutions is accessible in $[x2,x3]\times[y2,y3]$ only if $x_0 \in [x2,x3]\times[y2,y3]$.\\

\paragraph{\textbf{Consequences}}

Let $P_2(x_0)=0$ with $x_0 \in [x2,x3]$, then:
$$
\left \{
\begin{array}{ll}
yP_1(x_0) + P_0(x_0) = 0 & (1)\\
x_0^2 Q_2(y) + x_0Q_1(y) + Q_0(y) = 0 & (2)
\end{array}
\right.
$$

\begin{enumerate}
\item if $P_1(x_0)=0$ and $P_0(x_0)\neq0$, no solution
\item if $P_1(x_0)=0$ and $P_0(x_0)=0$, eq.$(1)$ gives no constraint on y
\item if $P_1(x_0)\neq0$ then $y_0 = -\frac{P_0(x_0)}{P_1(x_0)}$, then sol = $(x_0,y_0)$ if $\left\{\begin{array}{l}(x_0,y_0) \in [x2,x3]\times[y2,y3]\\x_0^2 Q_2(y_0) + x_0Q_1(y_0) + Q_0(y_0) = 0 \end{array}\right.$
\end{enumerate}

In case 2, we have to use eq. $(2)$ to get a value for y, which will give the same three possibilities (it is a degree 1 polynomial in y):
\begin{enumerate}
\item No solution if the coef. of y is zero but the rest is non-zero
\item All y are solution if the coef. of y is zero and the rest is zero too
\item There exist a unique solution $y_0 = -rest/coef.$, check whether it lies in $[y2,y3]$
\end{enumerate}


\newpage
\subsection{$Q_2(y) = 0$}

Similarly to the previous case, $Q_2(y) = 0$ is only possible if:
\begin{enumerate}
\item If $\frac{B_{y,0}-B_{y,1}}{x3-x1} + \frac{B_{y,2}-B_{y,1}}{x4-x2} \neq 0$, no possible solution
\item If $\frac{B_{y,0}-B_{y,1}}{x3-x1} + \frac{B_{y,2}-B_{y,1}}{x4-x2} = 0$, all y are solution
\item Otherwise $y_0 = -\frac{\frac{B_{y,0}-B_{y,1}}{x3-x1} + \frac{B_{y,2}-B_{y,1}}{x4-x2}}{\frac{A_{y,0}-A_{y,1}}{x3-x1} + \frac{A_{y,2}-A_{y,1}}{x4-x2}}$ is the unique solution
\end{enumerate}

And if there is a solution $y_0 \in [y2,y3]$, then:
\begin{enumerate}
\item if $Q_1(y_0)=0$ and $Q_0(y_0)\neq0$, no solution
\item if $Q_1(y_0)=0$ and $Q_0(y_0)=0$, eq.$(2)$ gives no constraint on x
\item if $Q_1(y_0)\neq0$ then $x_0 = -\frac{Q_0(y_0)}{Q_1(y_0)}$, then sol = $(x_0,y_0)$ if $\left\{\begin{array}{l}(x_0,y_0) \in [x2,x3]\times[y2,y3]\\y_0^2 P_2(x_0) + y_0P_1(x_0) + P_0(x_0) = 0 \end{array}\right.$
\end{enumerate}

In case 2, we have to use eq. $(1)$ to get a value for x, which will give the same three possibilities (it is a degree 1 polynomial in y):
\begin{enumerate}
\item No solution if the coef. of x is zero but the rest is non-zero
\item All y are solution if the coef. of x is zero and the rest is zero too
\item There exist a unique solution $x_0 = -rest/coef.$, check whether it lies in $[x2,x3]$
\end{enumerate}

\newpage
\subsection{$P_2(x)\neq0 \text{ and } Q_2(y)\neq0$}

In this general case, we have to check the existence of solutions to two degree 2 polynomials:
$$
\begin{array}{ll}
& \left \{
\begin{array}{ll}
\Delta_y(x) = P_1^2(x) - 4P_2^2(x)P_0^2(x) \geq 0 & (1)\\
\Delta_x(y) = Q_1^2(y) - 4Q_2^2(y)Q_0^2(y) \geq 0 & (2)
\end{array}
\right.\\
\Leftrightarrow & \left \{
\begin{array}{ll}
x^2 \left( a_{P1}^2-4a_{P2}a_{P0} \right) + x\left( 2a_{P1}b_{P1} - 4a_{P2}b_{P0} - 4a_{P0}b_{P2} \right) + b_{P1}^2-4b_{P2}b_{P0} \geq 0 & (1)\\
y^2 \left( a_{Q1}^2-4a_{Q2}a_{Q0} \right) + y\left( 2a_{Q1}b_{Q1} - 4a_{Q2}b_{Q0} - 4a_{Q0}b_{Q2} \right) + b_{Q1}^2-4b_{Q2}b_{Q0} \geq 0 & (2)
\end{array}
\right.
\end{array}
$$

Which means, by defining $\left\{\begin{array}{l}\delta_x = \left( 2a_{P1}b_{P1} - 4a_{P2}b_{P0} - 4a_{P0}b_{P2} \right)^2-4\left( a_{P1}^2-4a_{P2}a_{P0} \right)\left(b_{P1}^2-4b_{P2}b_{P0}\right)\\
\delta_y = \left( 2a_{Q1}b_{Q1} - 4a_{Q2}b_{Q0} - 4a_{Q0}b_{Q2} \right)^2-4\left( a_{Q1}^2-4a_{Q2}a_{Q0} \right)\left(b_{Q1}^2-4b_{Q2}b_{Q0}\right)\end{array}\right.$:
\begin{enumerate}
\item If $\left\{\begin{array}{l}a_{P1}^2-4a_{P2}a_{P0}=0\\ 2a_{P1}b_{P1} - 4a_{P2}b_{P0} - 4a_{P0}b_{P2} > 0\end{array}\right.$ then $x \in [x_0,\infty]\cap[x2,x3]$, with $x_0=-\frac{b_{P1}^2-4b_{P2}b_{P0}}{2a_{P1}b_{P1} - 4a_{P2}b_{P0} - 4a_{P0}b_{P2}}$
\item If $\left\{\begin{array}{l}a_{P1}^2-4a_{P2}a_{P0}=0\\ 2a_{P1}b_{P1} - 4a_{P2}b_{P0} - 4a_{P0}b_{P2} < 0\end{array}\right.$ then $x \in [-\infty,x_0]\cap[x2,x3]$, with $x_0=-\frac{b_{P1}^2-4b_{P2}b_{P0}}{2a_{P1}b_{P1} - 4a_{P2}b_{P0} - 4a_{P0}b_{P2}}$
\item If $\left\{\begin{array}{l}a_{P1}^2-4a_{P2}a_{P0}=0\\ 2a_{P1}b_{P1} - 4a_{P2}b_{P0} - 4a_{P0}b_{P2} = 0\end{array}\right.$ then $\left\{\begin{array}{l}b_{P1}^2-4b_{P2}b_{P0}=0 \Rightarrow \text{ all x solutions}\\ b_{P1}^2-4b_{P2}b_{P0} \neq 0 \Rightarrow \text{ no solution}\end{array}\right.$
\item If $\left\{\begin{array}{l}\delta_x < 0\\a_{P1}^2-4a_{P2}a_{P0} > 0 \end{array}\right.$, then all $x \in [x2,x3]$ are solution
\item If $\left\{\begin{array}{l}\delta_x < 0\\a_{P1}^2-4a_{P2}a_{P0} < 0 \end{array}\right.$, then no solution
\item If $\left\{\begin{array}{l}\delta_x \geq 0\\a_{P1}^2-4a_{P2}a_{P0} > 0 \end{array}\right.$, then $x \in \left([-\infty,x_1] \cup [x_2,\infty]\right) \cap [x2,x3]$ with $x_{1,2}=\frac{-\left( 2a_{P1}b_{P1} - 4a_{P2}b_{P0} - 4a_{P0}b_{P2} \right) \pm \sqrt{\delta_x}}{2\left( a_{P1}^2-4a_{P2}a_{P0} \right)}$
\item If $\left\{\begin{array}{l}\delta_x \geq 0\\a_{P1}^2-4a_{P2}a_{P0} < 0 \end{array}\right.$, then $x \in [x_1,x_2] \cap [x2,x3]$ with $x_{1,2}=\frac{-\left( 2a_{P1}b_{P1} - 4a_{P2}b_{P0} - 4a_{P0}b_{P2} \right) \pm \sqrt{\delta_x}}{2\left( a_{P1}^2-4a_{P2}a_{P0} \right)}$
\end{enumerate}

And idem for y

\newpage
\begin{landscape}
\paragraph{\textbf{Consequences}} Now let $I_x = [x_1,x_2] \subset [x2,x3]$ and $I_y = [y_1,y_2] \subset [y2,y3]$ be the two intervals satisfying $\Delta_y(x)\geq0$ and $\Delta_x(y)\geq0$, then:
$$
\begin{array}{ll}
&
\left \{
\begin{array}{l}
y = \frac{-P_1(x) \pm \sqrt{\Delta_y(x)}}{2P_2(x)}\\
x = \frac{-Q_1(y) \pm \sqrt{\Delta_x(y)}}{2Q_2(y)}
\end{array}
\right.\\
\Leftrightarrow &
\left \{
\begin{array}{l}
2yP_2\left( \frac{-Q_1(y) \pm \sqrt{\Delta_x(y)}}{2Q_2(y)} \right) = -P_1\left( \frac{-Q_1(y) \pm \sqrt{\Delta_x(y)}}{2Q_2(y)} \right) \pm \sqrt{\Delta_y\left( \frac{-Q_1(y) \pm \sqrt{\Delta_x(y)}}{2Q_2(y)} \right)}\\
2xQ_2\left( \frac{-P_1(x) \pm \sqrt{\Delta_y(x)}}{2P_2(x)} \right) = -Q_1\left( \frac{-P_1(x) \pm \sqrt{\Delta_y(x)}}{2P_2(x)} \right) \pm \sqrt{\Delta_x\left( \frac{-P_1(x) \pm \sqrt{\Delta_y(x)}}{2P_2(x)} \right)}
\end{array}
\right.\\
\Leftrightarrow &
\left \{
\begin{array}{l}
2y\left( a_{P2}\left( -Q_1(y) \pm \sqrt{\Delta_x(y)} \right) + 2Q_2(y)b_{P2} \right) = -\left( a_{P1} \left(-Q_1(y) \pm \sqrt{\Delta_x(y)} \right) + 2Q_2(y)b_{P1} \right) \pm 2Q_2(y)\sqrt{\Delta_y\left( \frac{-Q_1(y) \pm \sqrt{\Delta_x(y)}}{2Q_2(y)} \right)}\\
2x\left( a_{Q2}\left(-P_1(x) \pm \sqrt{\Delta_y(x)} \right) + 2P_2(x)b_{Q2} \right) = -\left( a_{Q1} \left(-P_1(x) \pm \sqrt{\Delta_y(x)} \right) + 2P_2(x)b_{Q1} \right) \pm 2P_2(x)\sqrt{\Delta_x\left( \frac{-P_1(x) \pm \sqrt{\Delta_y(x)}}{2P_2(x)} \right)}
\end{array}
\right.\\
\Leftrightarrow &
\left \{
\begin{array}{l}
2y\left( a_{P2}\left( -a_{Q1}y - b_{Q1} \pm \sqrt{\Delta_x(y)} \right) + 2b_{P2}\left(a_{Q2}y+b_{Q2}\right) \right) = -\left( a_{P1} \left(-a_{Q1}y - b_{Q1} \pm \sqrt{\Delta_x(y)} \right) + 2b_{P1}\left( a_{Q2}y+b_{Q2}\right) \right) \pm 2Q_2(y)\sqrt{\Delta_y\left( \frac{-Q_1(y) \pm \sqrt{\Delta_x(y)}}{2Q_2(y)} \right)}\\
2x\left( a_{Q2}\left(-a_{P1}x-b_{P1} \pm \sqrt{\Delta_y(x)} \right) + 2b_{Q2}\left(a_{P2}x+b_{P2}\right) \right) = -\left( a_{Q1} \left(-a_{P1}x-b_{P1} \pm \sqrt{\Delta_y(x)} \right) + 2b_{Q1}\left(a_{P2}x+b_{P2}\right) \right) \pm 2P_2(x)\sqrt{\Delta_x\left( \frac{-P_1(x) \pm \sqrt{\Delta_y(x)}}{2P_2(x)} \right)}
\end{array}
\right.\\
\Leftrightarrow &
\left \{
\begin{array}{l}
y^2 \left( -2a_{P2}a_{Q1}+4b_{P2}a_{Q2} \right) + y\left( -2a_{P2}b_{Q1}+2b_{P2}b_{Q2}-a_{P1}a_{Q1}+2b_{P1}a_{Q2} \pm 2a_{P2}\sqrt{\Delta_x(y)} \right) + \left( -a_{P1}b_{Q1} + 2b_{P1}b_{Q2} \pm a_{P1}\sqrt{\Delta_x(y)} \right) \mp 2Q_2(y)\sqrt{\Delta_y\left( \frac{-Q_1(y) \pm \sqrt{\Delta_x(y)}}{2Q_2(y)} \right)}\\
x^2 \left( -2a_{Q2}a_{P1}+4b_{Q2}a_{P2} \right) + x\left( -2a_{Q2}b_{P1}+2b_{Q2}b_{P2}-a_{Q1}a_{P1}+2b_{Q1}a_{P2} \pm 2a_{Q2}\sqrt{\Delta_y(x)} \right) + \left( -a_{Q1}b_{P1} + 2b_{Q1}b_{P2} \pm a_{Q1}\sqrt{\Delta_y(x)} \right) \mp 2P_2(x)\sqrt{\Delta_x\left( \frac{-P_1(x) \pm \sqrt{\Delta_y(x)}}{2P_2(x)} \right)}
\end{array}
\right.
\end{array}
$$


... And this is as far as I get analytically, any idea ? any reference ?

\end{landscape}


%%% End document
\end{document}
